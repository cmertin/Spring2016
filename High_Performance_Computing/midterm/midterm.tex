\documentclass[addpoints]{exam}
\usepackage{amsmath,amsthm,amssymb,url}

\usepackage{algorithm}
\usepackage{algorithmic}
\usepackage{graphicx}
\usepackage{float}
\usepackage{units}
%\usepackage{algpseudocode}

\usepackage[pdftex]{hyperref}
\usepackage{tikz}
\usepackage{pgfplots}
\usetikzlibrary{calc,3d,decorations.markings, backgrounds, positioning,intersections,shapes}

\renewcommand{\algorithmicrequire}{\textbf{Input:}}
\renewcommand{\algorithmicensure}{\textbf{Output:}}
%\algnewcommand\algorithmicparfor{\textbf{parallel for}}
%\algdef{S}[FOR]{PFOR}[1]{\algorithmicparfor\ #1\ \algorithmicdo}

\newcommand{\BigO}[1]{\mathcal{O}\left( #1\right)}
\newcommand{\ceil}[1]{\left \lceil #1 \right \rceil }
\newcommand{\floor}[1]{\left \lfloor #1 \right \rfloor }
\newcommand{\Mod}[1]{\ \text{\bf mod}\ #1}

\newtheorem{lemma}{Lemma}[section]
\newcommand{\var}{\text{Var}}
\title{CS 6230: Midterm}
\date{Due Date: March 31, 2016}
\author{Christopher Mertin}
\begin{document}
\maketitle
%\begin{center}
%\fbox{\fbox{\parbox{5.5in}{\centering
%This assignment has \numquestions\ questions, for a total of \numpoints\
%points.
%Unless otherwise specified, complete and reasoned arguments will be
%expected for all answers. 
%}}}
%\end{center}

\qformat{Question \thequestion: \thequestiontitle\dotfill}
\pointname{}
\bonuspointname{}
\pointformat{[\bfseries\thepoints]}

\printanswers



\begin{questions}

\titledquestion{Parallel Maximum Subarray}
You are given a one dimensional array that may contain both positive and negative numbers, develop a parallel algorithm to find the sum of contiguous subarray of numbers which has the largest sum. For example, if the given array is $[-2,-5,\underline{6},\underline{-2},\underline{-3},\underline{1},\underline{5},-6]$, then the maximum subarray sum is $7$ (underlined numbers).

\begin{solution}
The sequential version of this is known as \href{https://en.wikipedia.org/wiki/Maximum_subarray_problem}{Kadane's Algorithm} for the maximum subarray problem and, and the pseudocode is given as

\begin{algorithm}[H]
\caption{Kadane's Algorithm}
\begin{algorithmic}[1]
\REQUIRE{Array $A\in \mathbb{R}^{n\times 1}$}
\ENSURE{Maximum sum of contiguous subarray}
\STATE{$max_{1}\leftarrow 0$}
\STATE{$max_{2}\leftarrow 0$}
\FOR{$i = 0$ \TO $n-1$}
    \STATE{$max_{2}\leftarrow max_{1} + A_{i}$}
    \IF{$max_{2} < 0$}
       \STATE{$max_{2}\leftarrow 0$}
    \ENDIF
    \IF{$max_{1} < max_{2}$}
       \STATE{$max_{1}\leftarrow max_{2}$}
    \ENDIF
\ENDFOR
\RETURN{$max_{1}$}
\end{algorithmic}
\end{algorithm}

which is $\BigO{n}$ in complexity. This can be done in parallel in $\BigO{\log^{2}(n/p)}$ in the following way, where a Reduction would be performed at {\em each level} of the Prefix Sum, hence it's denoted as ``PrefixSumReduction.''

\begin{algorithm}[H]
\caption{Parallel Max Subarray Problem}
\begin{algorithmic}[1]
\REQUIRE{Array $A\in \mathbb{R}^{n\times 1}$}
\ENSURE{Maximum sum of contiguous subarray}
\STATE{$b_{1}\leftarrow$ PrefixSumReduction($A$,{\sc Max})}
\COMMENT{PrefixSum is inclusive}
\STATE{$b_{2}\leftarrow $ PrefixSumReduction($A$,{\sc Max})}
\COMMENT{PrefixSum is exclusive}
\RETURN{{\sc Max}($b_{1}$,$b_{2}$)}
\end{algorithmic}
\end{algorithm}
\end{solution}
\ \newpage
\titledquestion{Parallel Array Reordering}
Suppose we are given a set of $n$ elements stored in an array $A$ together with an array $L$ such that $L_{i} \in \{1,2,\ldots,k\}$ represents the label of element $A_{i}$, where $k$ is a constant. Develop an optimal $\BigO{\log(n)}$ time {\sc Er--Ew Pram} algorithm that stores all the elements of $A$ with label $1$ into the upper part of $A$ while preserving their initial ordering, followed by the elements labeled $2$, with the same initial ordering, and so on. For Example:

\begin{align*}
A &= \left[ 6, 5, 3, 9, 11, 12, 8, 17, 21, 2\right]\\
L &= \left[ 1, 1, 2, 3, \phantom{1}2, \phantom{1}1, 1, \phantom{1}2, \phantom{1}3, 3\right]\\
\intertext{produces}
A &= \left[ 6,5,12,8,3,11,17,9,21,2 \right]
\end{align*}

\begin{solution}
A sequential version of this code would be as follows

\begin{algorithm}[H]
\caption{Sequential Array Reordering}
\begin{algorithmic}[1]
\REQUIRE{$A\in \mathbb{R}^{n\times 1}$, $L\in \mathbb{R}^{n\times 1}$}
\ENSURE{$C\in \mathbb{R}^{n\times 1}$ as a sorted array according to $L$}
\STATE{Create $M\in \mathbb{R}^{k\times n}$}
\FOR{$i = 0$ \TO $n-1$}
    \STATE{$j\leftarrow L_{i}$}
    \STATE{$M_{j,i}\leftarrow A_{i}$}
\ENDFOR
\STATE{$C \leftarrow $ In-Order Traversal of elements in $M$}
\RETURN{$C$}
\end{algorithmic}
\end{algorithm}

This would be done in $\BigO{n}$. In parallel, for a {\sc Er--Ew Pram} model, to do this in $\BigO{\log(n)}$ can be done with the following algorithm, provided that $p > \frac{n}{\log (n)}$

\begin{algorithm}[H]
\caption{Parallel Array Reordering}
\begin{algorithmic}[1]
\REQUIRE{$A\in \mathbb{R}^{n\times 1}$, $L\in \mathbb{R}^{n\times 1}$}
\ENSURE{$C\in \mathbb{R}^{n\times 1}$ as a sorted array according to $L$}
\STATE{$\ell_{max} \leftarrow $ Reduction$(L,${\sc Max}$)$}
\FOR{$i = 0$ \TO $\ell_{max}$}
\STATE{$S \leftarrow \{\}$}
\STATE{$\ell \leftarrow $Reduction$(L,${\sc Min}$+i)$}
\FOR{$j = \frac{n}{p}(thread)$ \TO $\frac{n}{p}(thread + 1)$}
    \IF{$L_{j} = \ell$}
        \STATE{$S \leftarrow $ Append$(A_{j})$}
    \ENDIF
\ENDFOR
\STATE{Gather$(C,S)$}
\COMMENT{Gathers all arrays of $S$ to $C$}
\ENDFOR
\RETURN{$C$}
\end{algorithmic}
\end{algorithm}
\end{solution}

\ \newpage
\titledquestion{Parallel Fibonacci Numbers}
Develop a work and depth optimal parallel algorithm for computing the first $n$ \href{https://en.wikipedia.org/wiki/Fibonacci_number}{Fibonacci Numbers}

\begin{solution}
There are two sequential implementations for calculating the Fibonacci Sequence. Algorithm~\ref{alg:fib1} is the brute force approach, which is just based on the definition of calculating the Fibonacci Sequence. The time complexity of this algorithm is $T(n) = T(n-1) + T(n-2)$, which is exponential in $n$.

On the other hand, Algorithm~\ref{alg:fib2} is an approach that is faster but trades off with utilizing storage. Instead of recalculating {\em every} Fibonacci number, Algorithm~\ref{alg:fib2} stores the ones that have been calculated already. It is therefore a much more efficient algorithm, though it does take up $\BigO{n}$ storage. The time complexity for this algorithm is only $\BigO{n}$ which is much better than Algorithm~\ref{alg:fib1}, so it would be more ideal to parallelize this version to get a larger benefit.

\begin{algorithm}[H]
\centering
\caption{$Fibonacci(n)$}
\begin{algorithmic}[1]
\REQUIRE{$n^{th}$ Fibonacci Number you're calculating}
\ENSURE{Value of $n^{th}$ Fibonacci Number}
\IF{$n = 0$}
   \RETURN{0}
\ELSIF{$n = 0$ \OR $n = 2$}
       \RETURN{1}
\ELSE
       \RETURN{$Fibonacci(n-1)+Fibonacci(n-2)$}
\ENDIF
\end{algorithmic}
\label{alg:fib1}
\end{algorithm}

\begin{algorithm}[H]
\centering
\caption{$Fibonacci(A,n)$}
\begin{algorithmic}[1]
\REQUIRE{Array $A\in \mathbb{R}^{n\times 1}$, $n^{th}$ Fibonacci Number you're calculating}
\ENSURE{Value of $n^{th}$ Fibonacci Number}
\IF{$n = 0$}
   \RETURN{0}
\ELSIF{$n = 1$ \OR $n = 2$}
   \STATE{$A_{n}\leftarrow 1$}
   \RETURN{1}
\ELSIF{$A_{n} \neq 0$}
   \RETURN{$A_{n}$}
\ELSE
   \STATE{$A_{n} \leftarrow Fibonacci(A,n-1)+Fibonacci(A,n-2)$}
   \RETURN{$A_{n}$}
\ENDIF
\end{algorithmic}
\label{alg:fib2}
\end{algorithm}

\begin{algorithm}[H]
\centering
\caption{(Parallel) $Fibonacci(A,n)$}
\begin{algorithmic}[1]
\REQUIRE{Array $A\in \mathbb{R}^{n\times 1}$, $n^{th}$ Fibonacci Number you're calculating}
\ENSURE{Value of $n^{th}$ Fibonacci Number}
\IF{$n = 0$}
   \RETURN{0}
\ELSIF{$n = 1$ \OR $n = 2$}
   \STATE{$A_{n}\leftarrow 1$}
   \RETURN{1}
\ELSIF{$A_{n} \neq 0$}
   \RETURN{$A_{n}$}
\ELSE
   \STATE{$A_{n}^{(1)}\leftarrow Fibonacci(A,n-1)$}
   \COMMENT{Done by original thread}
   \STATE{$A_{n}^{(2)}\leftarrow Fibonacci(A,n-2)$}
   \COMMENT{Given to a new thread to be done in parallel}
   \STATE{(Sync the threads)}
   \RETURN{$A_{n}^{(1)} + A_{n}^{(2)}$}
   \COMMENT{From the original thread}
\ENDIF
\end{algorithmic}
\label{alg:fib2par}
\end{algorithm}

Since this is being done in parallel, it the complexity would reduce down to $\BigO{n/p}$.
\end{solution}

\ \newpage
\titledquestion{Distributed Matrix Transpose}
Suppose an $n\times n$ matrix is embedded in a hypercube (we assume that $n$ is a power of 2). Find an algorithm for transposing this matrix in $\BigO{\log(n)}$ time. 

{\em Hint:} $2^{k} = n^{2} = 2^{2q}$

\begin{solution}
The sequential case of Matrix Transpose takes $\BigO{n^{2}}$ time via the following algorithm

\begin{algorithm}[H]
\caption{Sequential Matrix Transpose}
\begin{algorithmic}[1]
\REQUIRE{Matrix $A\in \mathbb{R}^{n\times n}$}
\ENSURE{$A^{T}\in \mathbb{R}^{n\times n}$}
\FOR{$i = 0$ \TO $n-1$}
    \FOR{$j = 0$ \TO $n-1$}
        \STATE{$swap(A_{i,j},A_{j,i})$}
    \ENDFOR
\ENDFOR
\end{algorithmic}
\end{algorithm}

For the distributed case, we first have to define the set of diagonals. This problem {\em assumes} that each node contains one index of each matrix. Therefore, the node numbers is in the set $\{1,2,3,\ldots,n\times n\}$. Furthermore, we can define the diagonals of the matrix as being in the set $\{n\times n, n\times (n-1) - 1, n\times (n-2) - 2,\ldots,n\times 1 - (n-1)\}$. For the algorithm, these will be denoted as ``$Diag$.''

\begin{algorithm}[H]
\caption{Parallel Matrix Transpose}
\begin{algorithmic}
\REQUIRE{$nodeID$ (rank of each node), $n$}
\ENSURE{Transposed Matrix}
\IF{$nodeID \notin Diag$}
\STATE{$column \leftarrow \ceil{\frac{nodeID}{n}}$}
\STATE{$row \leftarrow nodeID \Mod{n}$}
\IF{$row = 0$}
   \STATE{$row \leftarrow n$}
\ENDIF
\STATE{$newColumn \leftarrow row$}
\STATE{$newRow \leftarrow row$}
\STATE{$otherNode \leftarrow n\times newColumn - (n - newRow)$}
\STATE{Send$(nodeID,otherNode)$}
\STATE{Receive$(otherNode,nodeID)$}
\ENDIF
\end{algorithmic}
\end{algorithm}

As the {\em maximum} communication time from one end of a hypercube to the other is $\BigO{log(n)}$, then the time complexity of this algorithm is $\BigO{log(n)}$. This above algorithm assumes that the node numbering is sequential along the row up through $n$. For example, the first column for $3\times 3$ is $\{1, 2, 3\}$, the second column is $\{4, 5, 6\}$, and the third is $\{7, 8, 9\}$.

\end{solution}


\ \newpage
\titledquestion{Parallel Horner's Algorithm}
Let $p(x) = a_{0}x^{n} + a_{1}x^{n-1} + \cdots + a_{n-1}x + a_{n}$ be a given polynomial. {\em Horner's Algorithm} can be used to compute $p(x)$ at a point $x_{0}$ is based on rewriting the expression for $p(x_{0})$ as follows:

\begin{align*}
p(x_{0}) &= \left( \cdots \left(\left( a_{0}x_{0} + a_{1}\right) x_{0} + a_{2}\right)x_{0} + \cdots + a_{n-1}\right)x_{0} + a_{n}
\end{align*}

An obvious sequential algorithm for this problem has $\BigO{n}$ complexity. Is it possible to develop a work optimal parallel algorithm whose complexity is $\BigO{n/p + \log(n)}$ for $p$ processors? Give pseudocode for the best possible parallel algorithm.

\begin{solution}
The sequential code that was discussed in the question can be seen in the algorithm below, which is $\BigO{n}$.
\begin{algorithm}[H]
\caption{Horner's Algorithm (Sequential)}
\begin{algorithmic}[1]
\REQUIRE{$a\in \mathbb{R}^{n\times 1}$, $x$}
\ENSURE{$p(x)$}
\STATE{$p\leftarrow 0$}
\FOR{$i = n$ \TO 0}
    \STATE{$p \leftarrow a_{i} + x\cdot p$}
\ENDFOR
\RETURN{$p$}
\end{algorithmic}
\end{algorithm}

The operation can be done in $\BigO{n/p + \log(n)}$, but it is not work optimal as it requires more work. In order to do so, take a look at the following example where $n = 40$ with $4$ nodes. This will put 10 on each node.

\begin{align*}
&\left(a_{39}x^{9}+a_{38}x^{8}+\cdots + a_{30}\right)x^{30}\\
&\left(a_{29}x^{9}+a_{28}x^{8}+\cdots + a_{20}\right)x^{20}\\
&\left(a_{19}x^{9}+a_{18}x^{18}+\cdots + a_{10}\right)x^{10}\\
&\left(a_{9}x^{9}+a_{8}x^{8} + \cdots + a_{1}x + a_{0}\right)
\end{align*}

From here, a reduction operation can be performed such to get the $\BigO{\log(n)}$ complexity, as the above gives $n/p$. This is not work optimal as you have to do more work, but the depth is reduced.

\begin{algorithm}[H]
\caption{Horner's Algorithm (Parallel)}
\begin{algorithmic}[1]
\REQUIRE{$a\in \mathbb{R}^{n\times 1}$, $x$}
\ENSURE{$p(x)$}
\STATE{$p \leftarrow 0$}
\IF{$thread \neq 1$}
\STATE{$high \leftarrow \frac{n}{p}(thread)$}
\STATE{$low \leftarrow \frac{n}{p}(thread - 1)$}
\STATE{$p \leftarrow \left(\sum_{i=low}^{high}a_{i}x^{i-low} \right)x^{low}$}
\ELSE
\STATE{$p \leftarrow \left( \sum_{i=n/p}^{1}a_{i}x^{(n/p)-i}\right)x + a_{0}$}
\ENDIF
\RETURN{Reduce($p$,{\sc Sum})}
\end{algorithmic}
\end{algorithm}
\end{solution}

\end{questions}
%\begin{thebibliography}{99}
%\bibitem{qselect}James Aspnes, Yale, \url{http://www.cs.yale.edu/homes/aspnes/pinewiki/QuickSelect.html}
%\end{thebibliography}

\end{document}

%%% Local Variables:
%%% mode: latex
%%% TeX-master: t
%%% End:
